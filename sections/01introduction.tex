\section{Introduction} \label{sec:outline}
    % Residual stresses evaluation based on acoustic signature of LSP

\subsection{Laser shock peening}

The configuration of an LSP process with a metallic component is shown in Figure~ \ref{fig:lspconfiguration}. An intense pulsed laser shock beam is focused onto a metal surface for a brief period of time (\SIrange{10}{100}{\ns}). The heated zone is vaporised and transformed to plasma by ionization (the temperature of plasma is over \SI{10000}{\degreeCelsius}). The plasma is under high pressure, which propagates through the material via shock waves. Two modes of LSP exist: 

\begin{itemize}

    \item direct ablation mode,
    \item confined ablation mode.

\end{itemize}

The direct ablation mode refers to the interaction of plasma with metal without coating and confinement \cite{ding_ye_2006}. Plasma pressure of tenths of a \SI{}{\GPa} is achieved using direct ablation mode. Higher pressures of \SIrange{1}{5}{\GPa} can be obtained using the confined mode. The confined ablation mode is known not only to increase the peak pressure of plasma by a factor up to ten, but also increases the duration of plasma by a factor of three in comparison with the direct ablation mode. In the confined mode, the metal surface is usually coated with an opaque material such as black paint or aluminium foil and confined by a material transparent to the laser radiation such as water or borosilicate glass. A stronger pressure pulse results in a higher magnitude of compressive residual stresses at a deeper depth \cite{fairland}. 

\begin{figure}[h]
    \centering
    \includegraphics[width=0.6\linewidth]{img/lsp_configuration.jpg}
    \caption{Schematic configuration of laser shock peening process \cite{fabbro_peyre_berthe_scherpereel_1998}}
    \label{fig:lspconfiguration}
\end{figure}

\subsection{Acoustic emission measurement}
Offline detection methods assess the process after it is carried out. On the other hand, online detection methods assess the process in real-time. Existing LSP detection methods (e.g. X-ray diffraction and hole-drilling stress measurements) are offline detection methods. The advantage of online detection methods is that errors in the process are immediately recognized and can be corrected. In order to overcome the disadvantages of offline detection methods, an online detection method based on LSP acoustic measurement is investigated in this section \cite{wu_zhao_qiao_liu_zhang_hu_2018}. An acoustic signal in the \SI{}{\milli\second}--range duration is emitted during the LSP process. By benchmarking the acoustic signal with an offline method, the correlation between the surface residual stress of the material and the acoustic signal can be revealed and unique signatures of acoustic signals for different materials and conditions can be identified \cite{banerjee_2019}. 

Laser processes monitoring is usually based on light-detecting sensors, such as photodiodes, cameras and spectrometers. An alternative method to monitor laser processes is to evaluate their sound and ultrasound emissions. Acoustic detectors are not widely used in process monitoring and control, mainly because of their limited frequency bandwidth and susceptibility to electromagnetic interference. The drawbacks of conventional capacitive microphones can be overcome by using an optical microphone \cite{fischer_rohringer_panzer_hecker_2017}. 